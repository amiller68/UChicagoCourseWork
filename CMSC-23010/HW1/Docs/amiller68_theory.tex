\documentclass[]{article}
%opening
\title{Theory}
\author{Alex Miller}
\addtolength{\oddsidemargin}{-.875in}
\addtolength{\evensidemargin}{-.875in}
\addtolength{\textwidth}{1.75in}

\addtolength{\topmargin}{-.875in}
\addtolength{\textheight}{1.75in}
\setcounter{secnumdepth}{0}
\usepackage{cancel}
\usepackage{amssymb}
\usepackage{amsmath}
\usepackage{hyperref}
\usepackage[ruled,vlined, linesnumbered]{algorithm2e}
\DontPrintSemicolon


\begin{document}
	\maketitle
	
\section{Theory Questions}
Provide solutions to questions 2, 3, 4, 5, 7, 8, 11, 14, 15, 16.

\subsection{Problem 2}
\begin{enumerate}
	\item Liveliness; the good thing that eventually happens is that all patrons will be served
	\item Safety; the bad thing that never happens is that things float away into the air
	\item Safety; the bad thing that never happens is that the program gets stuck
	\item Liveliness; the good thing that eventually happens is that the user is informed of the event
	\item Liveliness; the good thing that eventually happens is that the user is informed of the event
	\item Safety; the bad thing that never happens is that the cost of living decreases
	\item Liveliness; the good thing that eventually happens is that I will die and have to pay taxes 
	\item Safety; the bad thing that never happens is that I never get snuck up on by a Harvard man.
\end{enumerate}
\subsection{Problem 3}
Alice and Bob set cans up on each other's window sills; each one's can has a string attached to it that the other can use to knock it down.
Alice does the following:
\begin{enumerate}
	\item She waits until the can on her window sill is down.
	\item She releases the pets.
	\item When the pets return, Alice checks whether they finished the food. If so, she
	resets the can on her windowsill and then knocks down the can on Bob's window sill.
\end{enumerate}
Bob does the following:
\begin{enumerate}
	\item He waits until the can on his window sill is down.
	\item He puts food in the yard.
	\item He resets the can on his window sill and knocks down the can on Alice's window sill.
\end{enumerate}
\subsection{Problem 4}
The following assume $P \geq 2$
In the case that we know the initial state of the switch is off:
\begin{itemize}
	\item We designate one of the $P$ prisoners to be the $counter$; the counter maintains an internal count $c$. Initially, $c = 0$.
	\item When the counter enters the switch room:
	\begin{itemize}
		\item If the switch is down, they do nothing and leave
		\item If the switch is up, the counter increments $c$
	\end{itemize}
	\item When $c = P - 1$, the counter may declare that all $P$ prisoners have visited the switch room at least once
	\item For all other prisoners; when they enter the switch room:
	\begin{itemize}
		\item If the switch is down, and they have never flipped the switch before, they should switch it to the "on" position
		\item If the switch is up, they should do nothing and leave.
	\end{itemize}
\end{itemize}
In the case that we don't know the initial state of the switch:
\begin{itemize}
	\item We designate one of the $P$ prisoners to be the $counter$; the counter maintains an internal count $c$. Initially, $c = 0$.
	\item When the counter enters the switch room:
	\begin{itemize}
		\item If the switch is down, they do nothing and leave
		\item If the switch is up, the counter increments $c$
	\end{itemize}
	\item When $c = (2 * P) - 1$, the counter may declare that all $P$ prisoners have visited the switch room at least once
	\item For all other prisoners; when they enter the switch room:
	\begin{itemize}
		\item If the switch is down, and they have not flipped the switch more than twice, they should switch it to the "on" position
		\item If the switch is up, they should do nothing and leave.
	\end{itemize}
\end{itemize}
\subsection{Problem 5}

\subsection{Problem 7}
$S_n = \frac{1}{1 - p + \frac{p}{n}}$ \\
$S_2 = \frac{1}{1 - p + \frac{p}{2}}$ \\
$S_2 = \frac{1}{1 - \frac{p}{2}}$ \\
$1 - \frac{p}{2} = \frac{1}{S_2}$ \\
$p = 2 - \frac{2}{S_2}$ \\
$S_n = \frac{1}{1 - (2 - \frac{2}{S_2}) + \frac{(2 - \frac{2}{S_2})}{n}}$\\

\subsection{Problem 8}
\begin{itemize}
	\item Let $proc-a$ denote a uni-processor that executes five instructions per second
	\item Let $proc-b$ denote a ten processor multi-processor, each core of which executes 1 instruction per second
	\item Let $P$ be a program we are considering to run on either $proc-a$ or $proc-b$. The fraction of $P$ that is parallelizable is $p$.
	\item Let $A$ be the time it takes for $proc-a$ to run $P$
	\item Let $B_i, 0 \leq i \leq 10$ be the time it takes for $proc-b$ to run $P$ using $i$ cores
	\item $5 * A = B_1$
	\item $B_{10} = \frac{1}{1 - p + \frac{p}{10}} = \frac{1}{1 - \frac{9p}{10}}$
	\item Assume that $P$ runs faster on $proc-b$ than $proc-a$ while using all ten cores:
	\begin{itemize}
		\item That must mean that $B_{10}$ achieves a speedup ratio of $5$ or more in relation to $B_1$
		\item Therefore $B_{10} \geq 5$
		\item $\frac{1}{1 - \frac{9p}{10}} \geq 5$
		\item $\frac{9p}{10} \geq 1 - \frac{1}{5}$
		\item $\frac{9p}{10} \geq \frac{4}{5}$
		\item $9p \geq 8$
		\item $p \geq \frac{8}{9}$
	\end{itemize}
	\item Therefore, if $P$ runs better on $proc-b$ than $proc-a$, it must be that  $p \geq \frac{8}{9}$. 
	\item Therefore, if $p > \frac{8}{9}$, we should buy $proc-b$, and $proc-a$ if otherwise.
	\end{itemize}

\subsection{Problem 11}
\subsection{Problem 14}
\subsection{Problem 15}
\subsection{Problem 16}

\end{document}

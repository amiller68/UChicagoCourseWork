\documentclass[11pt]{article}
\usepackage{fullpage}
\usepackage{amsthm}

\usepackage{amsthm,amsmath,amsfonts,amssymb,amstext,enumitem}
\usepackage{latexsym,ifthen,url,rotating}

\usepackage[usenames,dvipsnames]{color}

% --- -----------------------------------------------------------------
% --- Document-specific definitions.
% --- -----------------------------------------------------------------

\newtheorem{problem}{Problem}
\newtheorem{theorem}{Theorem}
\newtheorem{definition}{Definition}
\newtheorem{lemma}{Lemma}

\newcommand{\bK}{\mathbf{K}}
\newcommand{\bM}{\mathbf{M}}
\newcommand{\bC}{\mathbf{C}}
\newcommand{\otp}{\mathrm{OTP}}
\newcommand{\msgs}{\mathcal{M}}
\newcommand{\ctxts}{\mathcal{C}}
\newcommand{\getsr}
  {{\:\stackrel{\raisebox{-2pt}{${\scriptscriptstyle \hspace{0.2em}\$}$}}
   {\leftarrow}\:}}
\newcommand{\points}[1]{\textbf{({#1} pts)}}

\newcommand{\Colon}{\ : \ }
\newcommand{\st}{\mathsf{state}}
\newcommand{\msgsp}{\mathcal{M}}
\newcommand{\keys}{\mathcal{K}}
\newcommand{\kg}{\mathcal{K}}
\newcommand{\enc}{\mathcal{E}}
\newcommand{\decc}{\mathcal{D}}
\newcommand{\MAC}{\mathrm{MAC}}
\newcommand{\RMAC}{\mathrm{RMAC}}

\newcommand{\pk}{pk}
\newcommand{\sk}{sk}

\newcommand{\AES}{\mathsf{AES}}

\newcommand{\algorithm}[1]{\textbf{Alg} {#1}}

\newcommand{\calO}{\mathcal{O}}

\newcommand{\dlog}{\mathrm{dlog}}

\newcommand{\Adv}{\mathbf{Adv}}
\newcommand{\AdvPRF}[2]{\Adv^{\mathrm{prf}}_{#1}({#2})}
\newcommand{\AdvCPA}[2]{\Adv^{\mathrm{ind{-}cpa}}_{#1}({#2})}
\newcommand{\AdvCCA}[2]{\Adv^{\mathrm{ind{-}cca}}_{#1}({#2})}
\newcommand{\AdvKR}[2]{\Adv^{\mathrm{kr}}_{#1}({#2})}
\newcommand{\AdvCKR}[2]{\Adv^{\mathrm{ckr}}_{#1}({#2})}
\newcommand{\AdvRMR}[2]{\Adv^{\mathrm{rmr}}_{#1}({#2})}
\newcommand{\AdvCR}[2]{\Adv^{\mathrm{cr}}_{#1}({#2})}
\newcommand{\AdvUFCMA}[2]{\Adv^{\textrm{uf{-}cma}}_{#1}({#2})}
\newcommand{\AdvDL}[2]{\Adv^{\mathrm{dl}}_{#1}({#2})}

\newcommand{\Exp}{\mathbf{Exp}}
\newcommand{\ExpOW}[1]{\Exp^{\mathrm{ow}}({#1})}
\newcommand{\ExpCKR}[2]{\Exp^{\mathrm{ckr}}_{#1}({#2})}
\newcommand{\ExpRMR}[2]{\Exp^{\mathrm{rmr}}_{#1}({#2})}

\newcommand{\concat}{{\,\|\,}}
\newcommand{\xor}{\oplus}
\newcommand{\bits}{\{0,1\}}

\newcommand{\tcolh}{T^{\mathrm{col}}_h}
\newcommand{\tcolH}{T^{\mathrm{col}}_{H^2}}
\newcommand{\Hcomb}{H^{1\|2}}
\newcommand{\Hxor}{H^{1\oplus2}}

\newcommand{\EXP}{\textrm{EXP}}
\newcommand{\MODEXP}{\textrm{MOD{-}EXP}}
\newcommand{\ADD}{\textrm{ADD}}
\newcommand{\MULTIMODEXP}{\textrm{MULTI{-}MOD{-}EXP}}
\newcommand{\MUL}{\textrm{MUL}}
\newcommand{\MOD}{\textrm{MOD}}

\newcommand{\GG}{\mathbb{G}}
\newcommand{\ZZ}{\mathbb{Z}}

% --- -----------------------------------------------------------------
% --- The document starts here.
% --- -----------------------------------------------------------------
\begin{document}
\sloppy

\noindent 
CMSC 28400: Introduction to Cryptography, Autumn 2021\\
University of Chicago\\
Course Staff: Akshima, David Cash (instructor), Andrew Chu, Alex Hoover

\begin{center}
\LARGE{\textbf{Problem Set 2}}\\
    \large{Due in Thursday, October 14 at 11:59pm}
\end{center}

\vspace{.1in}

\begin{description}

    \item[Collaboration policy.] Please respect the following collaboration
        policy:  You may discuss problems together in groups of up to four
        \emph{but you must write up your own solutions. You should never see
        your collaborators' papers or code.}  At the beginning of your
        submission, indicate the names of your (1, 2, or 3) collaborators, if
        any.   You may switch groups between problem sets but not within the
        same problem set.

    \item[Sources.] Cite any sources you use. Citing lecture, providing
    readings, or the other free textbooks linked from the course syllabus
    webpage is allowed. You may use results proved in those sources or in
    lecture without repeating their proofs.  Using Google, Chegg, or searching
    for posted solutions from other universities, is not allowed.

    \item[Grading.] Responses to theory problems will be graded for
        correctness, precision, \emph{and clarity}.  Simple, well-explained
        proofs are preferred.  Point values of problems vary, and are listed.  
        You are encouraged to ask for help and advice from the staff in writing
        up your solutions.

    \item[Submitting your solutions.] Solutions will be accepted on Gradescope.
    Check Campuswire for instructions.

    You may submit well-lit, clear photographs of handwritten solutions. You may
    also type up your solutions by modifying the included tex file -- The course
    staff will be happy to provide support if you choose to learn the Latex
    typesetting language.

\end{description}
\hrulefill

\begin{enumerate}

\item \points{50} The $n$-bit one-time pad $\otp_n$ may select the key
        $k=0^n$ (meaning the string of $n$ zeros), in which case encryption
        simply outputs the message $m$ without changing it.  A natural
        suggestion is to modify the scheme so that the key is never set to
        $0^n$. To avoid having fewer keys than messages, we also need to remove
        a message; Say we take out $m=0^n$ as well.

        More formally, define $\keys' = \msgs' = \bits^n\setminus\{0^n\}$,
        $\ctxts = \bits^n$, and $\otp'_n:\keys' \times \msgs' \to \ctxts$ by
        $\otp'_n(k,m) = k\oplus m$.

        Prove that $\otp'_n$ is not perfectly secret. 

\item \points{75} Fix some positive integer $n$.  Let $\keys$ be the set of
        all permutations on $\{1,\ldots,2n\}$.  Let $\msgs = \bits^n$, and
        $\ctxts = \bits^{2n}$. 

        For a bit-string $m\in\bits^n$, we write $\overline{m}$ for its
        \emph{bit-wise complement}, i.e. $m$ but with all of its bits flipped.
        So for example $\overline{0100} = 1011$.  We write $m_i$ for the $i$-th
        bit of $m$.

        Define $E:\keys\times\msgs\to\ctxts$ as follows. It takes as
        input a key $\pi\in\keys$ and $m\in\msgs$, and performs the following:
        \begin{enumerate}
            \item Let $r = m \| \overline{m}$, where $\|$ denotes string
                concatenation. Thus $r$ is $2n$ bits long.
            \item Define $c = r_{\pi(1)} r_{\pi(2)} \cdots r_{\pi(2n)}$. 
                Thus $c$ is also $2n$ bits long, and is equal to $r$ but
                with its bits permuted according to $\pi$ in the indicated
                manner.
            \item Output $c$.
        \end{enumerate}
        Prove that $E$ is perfectly secret. 

\item \points{100} Prove that a cipher $E:\keys\times\msgs\to\ctxts$ is
        perfectly secret if and only if it has independent ciphertexts.

\item \points{100} Consider the following definition:
        
             \begin{definition} 
                 Let $E:\keys\times\msgs\to\ctxts$ be a cipher and let $\bK$ be
                 a uniform random variable on $\keys$.  We say that $E$ has
                 \emph{uniform ciphertexts} if the following holds: For every
                 random variable $\bM$ on $\msgs$ that is independent of $\bK$,
                 the random variable $\bC = E(\bK,\bM)$ is uniform on $\ctxts$.
             \end{definition}

                    Prove or disprove the following:
        \begin{enumerate}
            \item If $E$ is perfectly secret then it has uniform ciphertexts.
            \item If $E$ has uniform ciphertexts then it is perfectly secret.
        \end{enumerate}

\end{enumerate}

\end{document}


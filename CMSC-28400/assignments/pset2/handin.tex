\documentclass[]{article}

%opening
\title{P-Set 2 Handin}
\author{Alex Miller}

\begin{document}

\maketitle

\section{}
\begin{itemize}
	\item Consider some $n \in Z, n > 2$ and $OPT'_n: K' x M' \leftarrow C$
	\item Consider two distinct messages $m_0, m_1 \in M'$, and a cipher text $c = m_0$
	\item Say we want to calculate the probabilities $Pr[OPT'_n(k, m_0) = c]$ and $ Pr[OPT'_n(k, m_1) = c]$, where $k$ is a uniform random variable on $K$
	\item There is (at least) one $k_1 \in K$ such that $k_1 \oplus m_1 = c$, since $m_1 \neq c$, meaning $k_1 \oplus m_1 = c$ for some non-zero string in $K$. Therefore $Pr[OPT'_n(k, m_1) = c] \geq \frac{1}{|K|}$
	\item However, $k_0 \notin K$ such that $k_0 \oplus m_0 = c$. This follows because $m_0 = c_0$, so $\{0\}^n \oplus m_o = c$, but $\{0\}^n \notin K$. Therefore $Pr[OPT'_n(k, m_0) = c] = 0$
	\item Therefore $Pr[OPT'_n(k, m_0) = c] \neq Pr[OPT'_n(k, m_1) = c]$, where $k$ is a uniform random variable on $K$
	\item $OTP'_n$ is not perfectly secret
	\item \textbf{QED}
	\iffalse
	\item $OTP'_1$ is not perfectly secret because it can only encode one message, and is therefore just plain not secret
	\item $OTP'_2$ is not perfectly secret:
	\begin{enumerate}
		\item Let $OPT'_2: K x M \leftarrow C$, where $K, M = {01, 10, 11}$ and $C = {00, 01, 10, 11}$
		\item Say we wanted to encrypt a message $m_0 = 01$ and $m_1 = 11$ using $OTP'_2$.
		\item Consider a $c \in C$, $c = 01$
		\item If $OTP'_2$ were perfectly secret, $m_0$ and $m_1$ would be encrypted to $c$ under $OTP'_2$ with equal probability.
		\item However, because $00 \notin K$, $m_0$ never gets encrypted to $c$ under $OTP'_2$, while $m_1$ can by $k = 10$
		\item Therefore $Pr[OPT'_2(k, m_0) = c] \neq Pr[OPT_n(k, m_1) = c]$, where $k$ is a uniform random variable on $K$
	\end{enumerate}
	\item If $OTP'_n$ isn't perfectly secret, neither is $OTP'_{n+1}$
	\begin{enumerate}
		\item 
	\end{enumerate}
	\fi
\end{itemize}
\section{}
\begin{enumerate}
	\item Consider some $r$. $r$ is a bit string of length $2n$. It contains $n$ '0's and $n$ '1's
	\begin{enumerate}
		\item $r$ is generated by concatenating a bit string $m$ of length $n$ to its complement $\lnot m$.
		\item Let the number of '0's in $m$  be $z$, and the number of '1's be $o$.
		\item $o = n - z$
		\item Let the number of '0's in $r$  be $z_r$, and the number of '1's be $o_r$.
		\item $o_r = z + o = z + n - z = n$
		\item Therefore $z_r = o_r = n$
	\end{enumerate}
	\item Consider some $c$ generated from $r$ by some $\pi in K$. $c$ also contains $n$ '0's and $n$ '1's since its generated by a permutation on the bits in $r$ by $\pi$.
	\item Therefore $\forall m \in M, \exists \pi \in K, s.t. E(\pi, m) = c$. In other words any pair of plain text messages $m_0, m_1 \in M$ can be encoded into one of the same set of cipher texts; $\{E(\pi, m_0): \pi \in K\} = \{E(\pi, m_1): \pi \in K\}$
	\item Moreover $\forall m_0, m_1 \in M, \forall c \in C, Pr[E(\pi, m_0) = c] = Pr[E(\pi, m_1) = c]$, where $\pi$ is a uniform random variable over $K$
	\begin{enumerate}
		\item Consider any pair of plain text messages $m_0, m_1$ and their corresponding intermediate values, $r_0 = m_0 || \lnot m_0 $ and  $r_1 = m_1 || \lnot m_1 $. 
		\item \textit{Case 1}: consider a cipher text $c$ s.t. $c \in \{E(\pi, m_0): \pi \in K\} = \{E(\pi, m_1): \pi \in K\}$. In other words, let $c$ be bit string of length $2n$ with $n$ '0's and $n$ '1's.
		\item Let $K_0 \subseteq K s.t. \forall \pi_0 \in K_0, E(\pi_0, m_0) = c$. $K_0$ describes all the distinct $\pi \in K$ that permute $r_0$ to $c$
		\item Let $K_1 \subseteq K s.t. \forall \pi_1 \in K_1, E(\pi_1, m_1) = c$. $K_1$ describes all the distinct $\pi \in K$ that permute to $r_1$ to $c$
		\item Given that $\pi$ is a uniform random variable over $K$, $Pr[E(\pi, m_0) = c] = \frac{|K_0|}{|K|}$ and $Pr[E(\pi, m_1) = c] = \frac{|K_1|}{|K|}$
		\item $|K_0| = |K_1|$
		\begin{enumerate}
			% Ersatz reasoning ... 
			\item For either $r_0$ and $r_1$, we can count $|K_0|$ and $|K_1|$ the same way.
			\item Imagine that we separate $r_0$ and $r_1$ into buckets of '0' and '1' bits. For each $r$-string, the '0' and '1' buckets each start with $n$ objectified bits.
			\item Consider the $2n$ bits of $c$ in order. Say that the first bit of $c$ is '0'; we would have $n$ ways of populating it from $r_0$'s '0'-bucket, and $n$ ways of populating it from $r_1$'s '0'-bucket.
			\item Moving down $c$, the next time we encounter a '0'-bit we would have $n -1$ remaining ways of populating it from $r_0$'s '0'-bucket, and $n - 1 $ ways of populating it from $r_1$'s '0'-bucket.
			\item Repeating this process for all $n$ '0'-bits in $c$, we can see that there are $n!$ ways of permuting the '0's in $r_0$ to the '0's in $c$. The same is true for the '0's in $r_1$
			\item We also have to account for '1'-bits in $c$. Applying the same reasoning we see that there are $n!$ ways of permuting the '1's in $r_0$ to the '1's in $c$. The same is true for the '1's in $r_1$
			\item Putting it all together, the total number of ways to permute $r_0$ to $c$ is $n!^2$, and the total number of ways to permute $r_1$ to $c$ is also $n!^2$
			\item These numbers describe $|K_0|$ and $|K_1$ respectively
			\item $|K_0| = |K_1|$
		\end{enumerate}
		\item \textit{Case 2}: Now consider a cipher text $c \in C$ s.t. $c \notin \{E(\pi, m_0): \pi \in K\} = \{E(\pi, m_1): \pi \in K\}$. 
		\item In the first case, $Pr[E(\pi, m_0) = c] = Pr[E(\pi, m_1) = c] = \frac{|K_0|}{|K|} = \frac{|K_1|}{|K|}$. In other words, $c$ can in fact be generated by $E$ and any message $m \in M$ gets encrypted into $c$ with some equal,non-zero probability.
		\item In the second case,  $c$ can't in fact be generated by $E$ and any message $m \in M$ never gets encrypted to $c$. Therefore $Pr[E(\pi, m_0) = c] = Pr[E(\pi, m_1) = c] = 0$
		\item In either case, $Pr[E(\pi, m_0) = c] = Pr[E(\pi, m_1) = c]$, where $\pi$ is a uniform random variable over $K$
	\end{enumerate}
	\item $E$ is therefore perfectly secret (Definition of Perfect Secrecy)
	\item \textbf{QED}
\end{enumerate}

\section{}
I looked in the textbook
\\
Consider a cipher $E: K$ X $M \leftarrow C$. Let the random variables \textbf{$k$} be uniformly distributed over $K$ and \textbf{$m$} over $M$.
\\
Let \textbf{$k$} and \textbf{$m$} be independent.
\\
Let $\textbf{c}$ define a random variable over $C$, $c := E(\textbf{k},\textbf{m})$
\\
First I will prove that if $E$ is perfectly secure, then it has independent cipher texts. In other words, I will show that if $E$ is perfectly secure, $\textbf{c}$ is independent of $\textbf{m}$
\begin{enumerate}
	\item We assume that $E$ is perfectly secure and consider any fixed $m \in M$ and $c \in C$
	\item We want to show that $Pr[\textbf{c} = c AND \textbf{m} = m] = Pr[\textbf{c}=c]Pr[\textbf{m}=m]$
	\item $Pr[\textbf{c} = c AND \textbf{m} = m]$ = $Pr[E(\textbf{k},\textbf{m})=cAND\textbf{m}=m]$
	\item = $Pr[E(\textbf{k},m)=cAND\textbf{m}=m]$
	\item $= Pr[E(\textbf{k},m)=c]Pr[\textbf{m}]$ (by independence of \textbf{m} and \textbf{k})
	\item Therefore we need to show that $Pr[E(\textbf{k},m)=c]=Pr[\textbf{c}=c]$. In other words, we can show that the probability of generating the cipher text $c$ is independent of $\textbf{m}$
	\item So starting from $c := E(\textbf{k},\textbf{m}$), $Pr[\textbf{c} = c ]=Pr[E(\textbf{k},\textbf{m}) =c]$
	\item = $\Sigma_{m' \in M} Pr[E(\textbf{k},\textbf{m}) =c AND \textbf={m} = m']$ (Law of total probability)
	\item = $\Sigma_{m' \in M} Pr[E(\textbf{k},m') =c AND \textbf={m} = m']$ 
	\item = $\Sigma_{m' \in M} Pr[E(\textbf{k},m') =c]Pr[\textbf={m} = m']$ (Independence of \textbf{k} and \textbf{k})
	\item = $\Sigma_{m' \in M} Pr[E(\textbf{k},m) =c]Pr[\textbf={m} = m']$ (Def of Perfect Secrecy)
	\item = $Pr[E(\textbf{k},m) =c]  \Sigma_{m' \in M} Pr[\textbf={m} = m']$
	\item = $Pr[E(\textbf{k},m) =c] $(probabilities sum to 1)
	\item \textbf{QED} 
\end{enumerate}

Next I will prove that if $E$ has independent cipher texts, it is perfectly secret. 
\begin{enumerate}
	\item Assume that \textbf{c} and \textbf{m} are independent and each message in M occurs with nonzero probability.
	\item Let $m \in M$, $c \in C$. It is enough to show from this that $Pr[\textbf{c} = c ]=Pr[E(\textbf{k},\textbf{m}) =c]$ to demonstrate perfect secrecy.
	\item $Pr[E(\textbf{k},m) =c]Pr[\textbf={m} = m] = Pr[E(\textbf{k},m) =c AND \textbf={m} = m]$ (\textbf{k,m} are independent, $Pr[\textbf=m]\neq 0$
	\item $ = Pr[E(\textbf{k},\textbf{m}) =c AND \textbf={m} = m]$
	\item $ =Pr[\textbf{c} = c  AND \textbf={m} = m]$
	\item $ =Pr[\textbf{c} = c]Pr[\textbf={m} = m]$ (by independence of $\textbf{c}$ and $\textbf{m}$)
\end{enumerate}

\textbf{A cipher $E: K x M \leftarrow C$ is perfectly secret $\Leftrightarrow \forall m_0, m_1 \in M, \forall c \in C, Pr[E(k, m_0) = c] = Pr[E(k, m_1) = c]$}
\\\\
\textbf{A cipher $E: K x M \leftarrow C$ has independent cipher texts if for all random variables $\Leftrightarrow \forall m_0, m_1 \in M, \forall c \in C, Pr[E(\pi, m_0) = c] = Pr[E(\pi, m_1) = c]$}
\\\\

\section{}
\subsection{a}
\begin{enumerate}
	\item Lets define a new cipher $E: K$ X $K'$ X $M \leftarrow C$ with the following properties:
	\begin{enumerate}
		\item $K = \{0, 1\}$, $K' = \{0, 1\}^2$, $M = \{0, 1\} ^ n$, $C = \{0, 1\} ^ {n + 1}$
		\item Define some function $OP:K' \leftarrow \{0, 1\}^1,$ and which maps two bit strings to one bit strings in a fixed but non-uniform grouping. For                                                              
		\item Given an $k \in K$, $k' \in K'$, $m \in M$, we calculate $c \in C$ accordingly:
		\\
		$c = E(k,k',m) = k \oplus m ||  OP(k')$
	\end{enumerate}
	\item $E$ is perfectly secret:
	\begin{enumerate}
		\item Say that $k$ is a uniform random variable on $K$. Being a bit string, we can also say that parts of $k$ are equivalently uniformly random on their sections. As such, we can say that the string $k_{n+1}||k_{n+2}$ is a uniform random variable on $\{0, 1\}^2$.
		\item Consider any message pair $m_0, m_1 \in M$, and any $c \in C$.
		\item Let $k$ and $k'$ be uniform random variables on the sets $K$ and $K'$, respectively
		\item Consider the first $n$ bits and last bit of our $c$. Call these $c'$ and $c''$, respectively.
		\item $Pr[k \oplus m_0 = c'] = Pr[k \oplus m_1 = c']$ (The One Time Pad is Perfectly Secret)
		\item Moreover, since the derivation of $c''$ is independent of either $m_0$ or $m_1$ we can further say that $Pr[E(k,k',m_0) = c' || c '' = c] = Pr[E(k,k',m_1) = c' || c '' = c]$
		\item Therefore $\forall m_0, m_1 \in M, \forall c \in C, Pr[E(k,k',m_0) = c] = Pr[E(k,k',m_1) = c]$ where $k$ and $k'$ are uniform random variables on $K$ and $K'$ respectively
	\end{enumerate}
	\item However, $E$ does  \textbf{not} have uniform cipher texts
	\begin{enumerate}
		\item Depending on how we choose to define $OP$, we can change the probabilities with which the values of $c''$ get encrypted by $E$. For example, if we make $OP$ an $OR$ operation on the bits in $k'$, $E$ encodes $c''$ to 1 with a bias of $.75$.
		\item Therefore, For every random variable $m$ on $M$, $k$ on $K$, $k'$ on $K'$, the random variable $C=E(K,M)$ is not uniform on the $Img(E)$  
	\end{enumerate}
	\item \textbf{QED}
\end{enumerate}
\subsection{b}
\begin{enumerate}
	\item Assume $E$ has uniform cipher texts and that it is not perfectly secret
	\item Then, for any random variable $m$ on $M$, $C = E(K,M)$ is uniform on $Img(E)$
	\item However, if $E$ wasn't perfectly secret, then $C$ wouldn't be uniform over $Img(E)$, because given a uniform $m$, the messages $M$ would map to cipher texts on $C$ with non uniform distribution.
	\item This is not the case, because we assume $E$ has uniform cipher texts.
	\item Therefore $E$ must be perfectly secret.
	\item If $E$ has uniform cipher texts it is perfectly secret
	\item \textbf{QED}

\end{enumerate}

\end{document}

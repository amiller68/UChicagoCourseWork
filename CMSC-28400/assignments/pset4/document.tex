\documentclass[]{article}
\usepackage[ruled,linesnumbered]{algorithm2e}
\DontPrintSemicolon
\usepackage[margin=0.5in]{geometry}

%opening
\title{Problem Set 4}
\author{Alex Miller}

\begin{document}
	
	\maketitle
	
\begin{abstract}
	Collaborators: Elizabeth Coble, Lucy Li
\end{abstract}



\section{}
Assume that each Attack has access to a (perfect) hash table $H$ of size $2^n$. Assume all the indices of $H$ are initialized to $\bot$. Assume that for all block ciphers, $F_1, F_2, F_3, and F_4$, we have $t$ examples of $(m, c)$ plaintext-ciphertext example pairs. Assume $t$ is sufficiently large to mitigate the probabilities of false positive keys. 
\subsection{$F_1: \{0,1\}^{3n}\times \{0,1\}^l \rightarrow \{0,1\}^l$}

$F_1(k_1 || k_2 || k_3, m) = E(k_3, E(k_2, E(k_1, m)))$
\\\\
$F_1^{-1}(k_1 || k_2 || k_3, c) = E^{-1}(k_1, E^{-1}(k_2, E^{-1}(k_3, c)))$
\\\\
\begin{algorithm}[H]
	\SetAlgoLined
	\For {$k \in \{0,1\}^n$} {
		$x \leftarrow E(k, m_1) || E(k, m_2) || ... || E(k, m_t)$\;
		$H[x] \leftarrow k$\;
	}

	\For {$k_2 || k_3 \in \{0,1\}^{2n}$} {
		$x \leftarrow = E^{-1}(k_2, E^{-1}(k_3, c_1)) || E^{-1}(k_2, E^{-1}(k_3, c_2)) || ... || E^{-1}(k_2, E^{-1}(k_3, c_t))$\;
		\If {$H[x] \neq \bot$} {
			$k_1 \leftarrow H[x]$\;
			return $k_1 || k_2 || k_3$\;
		}
	}
	\caption{$Attack((m_1,c_1),(m_2, c_2), ..., (m_t, c_t))$}
\end{algorithm}

This attack works because, for a given $(m, c)$ plaintext-ciphertext pair, the following should hold for the correct values of $k_1, k_2,$ and $k_3$:
\\\\
$E(k_1, m) = E^{-1}(k_2, E^{-1}(k_3, c))$
\\\\
This attack utilizes $2^n$ memory and executes in $2^{2n} + 2^n \approx 2^{2n}$ time

\subsection{$F_2: \{0,1\}^{n + l}\times \{0,1\}^l \rightarrow \{0,1\}^l$}

$F_2(k_1 || a_1, m) =  E(k_1, m \oplus a_1)$
\\\\
$F_2^{-1}(k_1 || a_1, c) =  E^{-1}(k_1, c) \oplus a_1$
\\\\
\begin{algorithm}[H]
	\SetAlgoLined
	\For {$k \in \{0,1\}^n$} {
		
		$x \leftarrow E^{-1}(k, c_1) || E^{-1}(k, c_2) || ... || E^{-1}(k, c_t)$\;
		$y \leftarrow m_1 || m_2 || ... || m_t$\;
		$z_1 || z_2 || ... || z_t \leftarrow x \oplus y$\;
		\If {$\forall i, j \in [t], i \neq j, z_i == z_j$} {
			return $k || z_1$
		} 
	}
	\caption{$Attack((m_1,c_1),(m_2, c_2), ..., (m_t, c_t))$}
\end{algorithm}
This attack works because, for a given $(m, c)$ plaintext-ciphertext pair, the following should hold for the correct values of $k_1$ and $a_1$:
\\\\
$m \oplus a_1 = E^{-1}(k_1, c)$, therefore $E^{-1}(k_1, c) \oplus m = a_1$. Therefore if we find a $k$ such that we can replicate some $a$ consistently (i.e. evaluate the same $a$ for all our examples using the same value of $k$) we know that $k = k_1$ and $a = a_1$.
\\\\ 
This attack utilizes minimal memory and takes $2^n$ time

\subsection{$F_3: \{0,1\}^{n + l}  \times \{0,1\}^l \rightarrow \{0,1\}^l$}

$F_3(k_1 || a_1, m) =  E(k_1, m) \oplus a_1$
\\\\
$F_3^{-1}(k_1 || a_1, c) =  E^{-1}(k_1, c \oplus a_1)$
\\\\
\begin{algorithm}[H]
	\SetAlgoLined
	\For {$k \in \{0,1\}^n$} {
		$x \leftarrow E(k, m_1) || E(k, m_2) || ... || E(k, m_t)$\;
		$H[x] \leftarrow k$\;
	}
	\For {$a \in \{0,1\}^{l}$} {
		$x \leftarrow c_1 \oplus a || c_2 \oplus a  || ... || c_t \oplus a $\;
		\If {$H[x] \neq \bot$} {
			$k \leftarrow H[x]$\;
			return $k || a$\;
		}
	}
	\caption{$Attack((m_1,c_1),(m_2, c_2), ..., (m_t, c_t))$}
\end{algorithm}
This attack works because, for a given $(m, c)$ plaintext-ciphertext pair, the following should for hold the correct values of $k_1$ and $a_1$:
\\\\
$E(k_1, m) = c \oplus a_1$
\\\\
This attack utilizes $2^n$ memory and takes $2^n + 2^l \approx 2^{l}$ time (assuming $n < l$)

\subsection{$F_4: \{0,1\}^{n + 2l}\times \{0,1\}^l \rightarrow \{0,1\}^l$}

$F_4(k_1 || a_1 || a_2, m) =  E(k_1, m \oplus a_1) \oplus a_2$
\\\\
$F_4^{-1}(k_1 || a_1 || a_2, c) =  E^{-1}(k_1, c \oplus a_2) \oplus a_1$
\\\\
\begin{algorithm}[H]
	\SetAlgoLined
	\For {$a \in \{0,1\}^l$} {
		$x \leftarrow m_1 \oplus a ||m_2 \oplus a  || ... || m_t \oplus a $\;
		$H[x] \leftarrow a$\;
	}
	
	\For {$k_1 || a_2 \in \{0,1\}^{n + l}$} {
		$x \leftarrow E^{-1}(k_1, c_1 \oplus a_2) || E^{-1}(k_1, c_2 \oplus a_2) || ... || E^{-1}(k_1, c_t \oplus a_2)$\;
		\If {$H[x] \neq \bot$} {
			$a_1 \leftarrow H[x]$\;
			return $k_1 || a_1 || a_2$\;
		}
	}
	\caption{$Attack((m_1,c_1),(m_2, c_2), ..., (m_t, c_t))$}
\end{algorithm}
This attack works because, for a given $(m, c)$ plaintext-ciphertext pair, the following should hold the correct values of $k_1, a_1$, and $a_2$:
\\\\
$m \oplus a_1 = E^{-1}(k_1, c \oplus a_2)$
\\\\
This attack utilizes $2^l$ space and takes $2^l + 2^{n+l} \approx 2^{n + l}$ time

\section{}
The probability that the attack generates a psuedo collision in the Hash table is $\frac{2^{112}}{2^{64t}}$:
\begin{itemize}
	\item Assume we've already populated our Hash table $H$ with $2^{56}$ distinct and uniform bit strings $\in \{0,1\}^{64t}$
	\item Now consider that we're starting our second for-loop, in which we will generate $2^{56}$ more uniformly random bit strings $\in \{0,1\}^{64t}$. We ask: "what is the probability that a string $x$ I generate in this step is already an index in $H$ ($Pr[H[x] \neq \bot]$)?"
	\item Well, $H$ contains $2^{56}$ examples from a uniformly random sample space of size $2^{64t}$; since $x$ is also uniformly random, $Pr[H[x] \neq \bot] = \frac{2^{56}}{2^{64t}}$ (the number of examples, divided by the size of the sample space)
	\item This tells us the probability that a single example $x$ generated by our second for-loop causes a collision in $H$. Call this event $E$. We want to know the probability that any $x$ generated by our second for-loop causes a collision in $H$. Call these set of events $\{E_1, E_2, ..., E_{2^{56}}\}$; we want to know $Pr[E_1 \cup E_2 \cup ... \cup E_{2^{56}}]$
	\item By applying Union Bound, we can say that $Pr[E_1 \cup E_2 \cup ... \cup E_{2^{56}}] \leq Pr[E_1] + Pr[E_2] + ... + Pr[E_{2^{56}}]$
	\item Therefore a good bound on the probability that an example of $x$ generated in our second step causes a collision in $H$ is $2^{56} * \frac{2^{56}}{2^{64t}} = \frac{2^{112}}{2^{64t}}$
	\item This offers itself as a good bound on the probability that our MITM attack generates a psuedo collision in $H$ while trying to determine the keys used in the Encryption scheme we're trying to break
\end{itemize}
This probability is less than 1 when $t = 2$
\\
This probability is less than $2^{-100}$ when $t = 4$

\section{}
An SPN scheme on messages of length $l$ with $r$ encryption rounds contains the following:
\begin{itemize}
	\item A key schedule comprised of $r + 1$ keys $\in \{0,1\}^l$. This set is not known to an adversary trying to break the SPN.
	\item A $t \times r$ table of S-boxes; $t * 8 = l$ (Each S-Box operates on one byte of data). This set is known to an adversary trying to break the SPN.
	\item A set of $r$ P-boxes. This set is known to an adversary trying to break the SPN.
\end{itemize}
When run conventionally, the key schedule of an SPN is hard to recover; an exhaustive key schedule search would take $2^{rl}$ time. However, given the proposed modification, there exists a much fast attack. Let's call the modified encryption scheme $SPN'$. Like the original $SPN$, $SPN'$ contains a key schedule $K$, a table of S-boxes $S$, and a set of P-boxes $P$. Unlike $SPN$, which computes a cipher text from a message by piping it through rounds of the transformations as described in $K$, $S$, and $P$, $SPN'$ works by:
\begin{enumerate}
	\item Performing all $r + 1$ key mixing transformations, as described in $K$
	\item Performing all $r$ S-box transformations, as described in $S$
	\item Performing all $r$ P-box transformations, as described in $P$
\end{enumerate}
We observe that, even knowing the contents of $S$ and $P$, $SPN$ is hard to crack because the contents of $K$ are unknown and, if chosen well, will intersperse new bits into cipher texts as they are permuted by the transformations in $S$ and $P$. Critically, we must realize two properties of $SPN'$, that will let us radically simplify our understanding of how $SPN'$ works:
\begin{enumerate}
	\item Because we apply all the key mixing steps in $K$ first, without permuting any bits between key mixing steps, we can understand the first step of $SPN'$ as applying a single key mixing step to our original message $m$ with a single key $k$, such that $k$ is equivalent to all the keys in $K$ xor'ed together.
	\item Because $SPN'$ doesn't intersperse bits between applications of transformations described in $S$ and $P$, we can understand next two steps of $SPN'$ as permuting the bits outputted by the first step of $SPN'$
\end{enumerate}
Therefore we can say that, given a message $m$, running $m$ through $SPN'$ (with a key schedule $K$, S-box table $S$, and P-Box set $P$) is equivalent to performing the following transformation on $m$:
\\\\
	$SPN'(m) = jumble(m \oplus k)$
\\\\
Where $k$ is the combined key, derived from $K$, and $jumble(x)$ is a function that permutes the bits in a bit-string $x$ and is equivalent to performing all the transformations described in $S$ and then all the transformations described in $P$ on the string $x$. 
\\\\
Since $S$ and $P$ are already known by an adversary, we can say that an adversary also knows the defintion of $jumlbe(x)$, as well as $jumble^{-1}(x)$; assume both of these functions can be run efficiently. Therefore, $SPN'$ can be broken by finding a usable key $k$ s.t. $k$ is equivalent to xor'ing together all the keys in the key schedule $K$. Assuming we have $t$ examples of $(m, c)$ plaintext-ciphertext examples under $SPN'$ with a key schedule $K$, we can feasibly attack $SPN'$ by performing an exhaustive key-search on the sample space of a single key in $K$, $\{0,1\}^l$:
\\
\begin{algorithm}[H]
	\SetAlgoLined
	\For {$k \in \{0,1\}^l$} {
		$x \leftarrow k \oplus m_1 || k \oplus m_2 || ... || k \oplus m_t$\;
		$y \leftarrow jumble^{-1}(c_1) || jumble^{-1}(c_2) || ... || jumble^{-1}(c_t)$\;
		\If {$x == y$} {
			return $k$
		}
	}
	\caption{$Attack((m_1,c_1),(m_2, c_2), ..., (m_t, c_t))$}
\end{algorithm}
We can say that our output $k$ is a (likely) correct (if $t$ is large enough), and can be used to decrypt cipher texts encrypted under our instance of $SPN'$
\\\\
Our attack therefore breaks $SPN'$ in time $2^l$ and utilizes minimal memory, which is far too easy given that our original key schedule sample space for $SPN$ was $2^{lr}$

\section{}
Here we consider how to break any 2-round $SPN$ with a 64-bit block size. Given such an $SPN$, let $SP_1(x)$ describe the process of running a 64-bit string $x$ though its first set of S-box and P-box steps, and $SP_2(x)$ describe the process of running a 64-bit string $x$ though its second set of S-box and P-box steps. Let $SP_1^{-1}(x)$ and  $SP_2^{-1}(x)$ describe the inverses of these processes, respectively. Assume that we can evaluate these processes efficiently for all $x \in \{0,1\}^{64}$.
\\\\
With this construction, we can describe the output of running a given 2-round, 64-bit SPN with a key schedule $K = \{k_1, k_2, k_3\}$ on any message $m$ as:
\\\\
$SPN(m) = SP_2(SP_1(m \oplus k_1) \oplus k_2) \oplus k_3$ and $SPN^{-1}(c) = SP_1^{-1}(SP_2^{-1}(c \oplus k_3) \oplus k_2) \oplus k_1$
\\\\
We will use this construction to show that the attacks we define below are sound. Assume that in either attack we have $t$ examples of $(m, c)$ plaintext-ciphertext examples under our given 2-round $SPN$ with key schedule $K$.
\subsection{}
Beginning with the construction of $SPN$ we sketched above, we observe that for a given $(m, c)$ plaintext-ciphertext pair in our example set, the following should hold for the correct values of $k_1, k_2$, and $k_3$:
\\\\
$SP_1(m \oplus k_1) = SP_2^{-1}(c \oplus k_3) \oplus k_2$
\\\\
Therefore, assuming that the three round keys are uniformly random and independent of one another, we can implement the following Meet-In-The-Middle attack to recover them. Assume that the attack has access to a (perfect) hash table $H$ of size $2^{64}$
\\\\
\begin{algorithm}[H]
	\SetAlgoLined
	\For {$k \in \{0,1\}^{64}$} {
		$x \leftarrow SP_1(m_1 \oplus k)  || SP_1(m_2 \oplus k)  || ... || SP_1(m_t \oplus k)$\;
		$H[x] \leftarrow k$\;
	}
	\For {$k_2 || k_3 \in \{0,1\}^{128}$} {
		$x \leftarrow SP_2^{-1}(c_1 \oplus k_3) \oplus k_2 || SP_2^{-1}(c_2 \oplus k_3) \oplus k_2  || ... || SP_2^{-1}(c_t \oplus k_3) \oplus k_2 $\;
		\If {$H[x] \neq \bot$} {
			$k_1 \leftarrow H[x]$\;
			return $k_1 || k_2 || k_3$\;
		}
	}
	\caption{$Attack((m_1,c_1),(m_2, c_2), ..., (m_t, c_t))$}
\end{algorithm}

This attack utilizes $2^{64}$ space and takes $2^{64} + 2^{128} \approx 2^{128}$ time
\subsection{}
If we can assume that the round keys are uniformly random but $k_1 = k_3$ (while $k_2$ remains independent) we can implement a much faster attack on our given 2-round SPN. For any $(m, c)$ plaintext-ciphertext pair in our example set, if we evaluate the following:
\\\\
$m' = SP_1(m \oplus k_1)$, $c' = SP_2^{-1}(c \oplus k_3)$
\\\\
We observe that:
\\\\
$m' \oplus k_2 = c'$, so $m' \oplus c' = k_2$
\\\\
In other words, if we find a $\hat{k}_1, \hat{k}_3$ such that we can replicate some $\hat{k}_2$ consistently (i.e. evaluate the same $\hat{k}_2$ for all our examples using the same values of $\hat{k}_1$ and $\hat{k}_3$) we know that $\hat{k}_1 = k_1$, $\hat{k}_2 = k_2$, and  $\hat{k}_3 = k_3$ (or, at the very least, these are defensible inductive guesses for the round keys, given2 our example set). Since we know that $k_1 = k_3$ we can run the following attack:
\\\\
\begin{algorithm}[H]
	\SetAlgoLined
	\For {$k \in \{0,1\}^{64}$} {
		$x \leftarrow SP_1(m_1 \oplus k) || SP_1(m_2 \oplus k) || ... || SP_1(m_t \oplus k)$\;
		$y \leftarrow SP_2^{-1}(c_1 \oplus k) || SP_2^{-1}(c_2 \oplus k) || ... || SP_2^{-1}(c_t \oplus k)$\;
		$z_1 || z_2 || ... || z_t \leftarrow x \oplus y$\;
		\If {$\forall i, j \in [t], i \neq j, z_i == z_j$} {
			return $k || z_1 || k$
		}
	}
	\caption{$Attack((m_1,c_1),(m_2, c_2), ..., (m_t, c_t))$}
\end{algorithm}
This attack uses minimal memory and takes $2^{64}$ time.

\section{}
We define the PRP advantage of an attack $A$ on a block-cipher $E: \{0,1\}^n \times \{0, 1\}^l \rightarrow \{0,1\}^l$ as:
\\\\
$Adv_E^{PRP}(A) = |Pr_K[A^{E(k, \cdot)} = 1] - Pr_\Pi[A^{\pi(\cdot)} = 1]|$
\\\\
where $k$ is a uniformly random variable on $K = \{0,1\}^n$ and $\pi$ is a uniformly random permutations on $\Pi = $ the set of permutations on bit strings of length $l$
\\\\
We define the following adversaries on our set of block ciphers, each built from a block cipher $E: \{0,1\}^n \times \{0, 1\}^l \rightarrow \{0,1\}^l$. Each adversary has oracle access $O$ to its respective block cipher or a random permutation.
\subsection{}
$F_1: \{0,1\}^n \times \{0,1\}^{2l} \rightarrow \{0,1\}^{2l}$
\\
$F_1(k, x) = E(k, x[1]) || x[2]$
\\\\
\begin{algorithm}[H]
	\SetAlgoLined
	$x \leftarrow O(\{1\}^l || \{0\}^l)$\;
	\If {$x[2] == \{0\}^l$} {
		return 1
	}
	return 0
	\caption{$A_1$}
\end{algorithm}
This adversary makes one query, with a constant runtime, assuming $E$ is efficient
\\\\
We can evaluate the following:
\\\\
$Pr_K[A_1^{F_1(k, \cdot)} = 1] = 1$: Because $F_1$ does not encrypt the second half of its input, we know if we call $O(x)$ s.t. $x \in \{0,1\}^{2l}, x[2] = \{0\}^l$, our oracle should return a $y$ s.t. $y \in \{0,1\}^{2l}, y[2] = \{0\}^l$, if the oracle was indeed submitting our request to $F_1$
\\\\
$Pr_\Pi[A_1^{\pi(\cdot)} = 1] = \frac{1}{2^l}$: The probability that submitting any input $x$ to the oracle and receiving an output $y$ s.t. $y \in \{0,1\}^{2l}, y[2] = \{0\}^l$ has a probability of $\frac{1}{2^l}$ (equivalent to the probability that any substring in $y$ is some fixed bit string of length $l$), if indeed the oracle was submitting our request to a random permutation $\pi$
\\\\
Therefore:
\\\\
$Adv_{F_1}^{PRP}(A_1) = |Pr_K[A_1^{F_1(k, \cdot)} = 1] - Pr_\Pi[A_1^{\pi(\cdot)} = 1]| = |1 - \frac{1}{2^l}| = 1 - \frac{1}{2^l}$
\subsection{}
$F_2: \{0,1\}^n \times \{0,1\}^{2l} \rightarrow \{0,1\}^{2l}$
\\
$F_2(k, x) = E(k, x[1]) || E(k, x[2])$
\\\\
\begin{algorithm}[H]
	\SetAlgoLined
	$x \leftarrow O(\{0\}^l || \{0\}^l)$\;
	\If {$x[1] == x[2]$} {
		return 1
	}
	return 0
	\caption{$A_2$}
\end{algorithm}
This adversary makes one query, with a constant runtime, assuming $E$ is efficient
\\\\
We can evaluate the following:
\\\\
$Pr_K[A_2^{F_2(k, \cdot)} = 1] = 1$: Because $F_2$ encrypts both halves its input using the same block cipher $E$ and key $k$, we know if we call $O(x)$ s.t. $x \in \{0,1\}^{2l}, x[1] == x[2]$, our oracle should return a $y$ s.t. $y \in \{0,1\}^{2l}, y[1] == y[2]$, if the oracle was indeed submitting our request to $F_2$
\\\\
$Pr_\Pi[A_2^{\pi(\cdot)} = 1] = \frac{1}{2^l}$: The probability that submitting any input $x$ to the oracle and receiving an output $y$ s.t. $y \in \{0,1\}^{2l}, y[1] == y[2]$ has a probability of $\frac{1}{2^l}$ (equivalent to the probability that any substring in $y$ is some fixed bit string of length $l$), if indeed the oracle was submitting our request to a random permutation $\pi$
\\\\
Therefore:
\\\\
$Adv_{F_2}^{PRP}(A_2) = |Pr_K[A_2^{F_2(k, \cdot)} = 1] - Pr_\Pi[A_2^{\pi(\cdot)} = 1]| = |1 - \frac{1}{2^l}| = 1 - \frac{1}{2^l}$
\subsection{}
$F_3: \{0,1\}^n \times \{0,1\}^{2l} \rightarrow \{0,1\}^{2l}$
\\
$F_3(k, x) = E(k, x[1] \oplus x[2]) || E(k, E(k, x[2]) \oplus x[1] \oplus x[2])$
\\\\
\begin{algorithm}[H]
	\SetAlgoLined
	$x \leftarrow  O(\{0\}^{2l})$\;
	$y \leftarrow  O(\{1\}^{2l})$\;
	\If {$x[1] == y[1]$} {
		return 1
	}
	return 0
	\caption{$A_3$}
\end{algorithm}
This adversary makes two queries, with a constant runtime, assuming $E$ is efficient
\\\\
We can evaluate the following:
\\\\
$Pr_K[A_3^{F_3(k, \cdot)} = 1] = 1$: The first half of any output $y$ from $F_3(k, x)$ is equivalent to $E(k, x[1] \oplus x[2])$. Therefore if we call $O(x)$ s.t. $x \in \{0,1\}^{2l}, x[1] == x[2]$, our oracle should return $E(k, x[1] \oplus x[2]) || z = E(k, \{0\}^l) || z$ (where $z \in \{0,1\}^l$) if it was indeed submitting our request to $F_3$. Therefore, if we submit two queries too our oracle s.t. for both inputs $x, y \in \{0,1\}^{2l}, x \neq y, x[1] == x[2], y[1] == y[2]$,  and receive outputs $a$ and $b$, then if our oracle submitted our requests to $F_3$ then $a[1] = b[1] = E(k, \{0\}^l)$
\\\\
$Pr_\Pi[A_3^{\pi(\cdot)} = 1] = \frac{1}{2^l}$: Say that our oracle was submitting queries to a random $\pi$ on $\Pi$, and that we feed it two valid queries, $x, y \in \{0,1\}^l$. Say we receive an output $a$ from our query $O(x)$, and $b$ from $O(y)$. Both $a$ and $b$ are random permutations. Let $a[1] = z$. The probability that $b[1] = z$ is $\frac{1}{2^l}$ (equivalent to the probability that any substring in $y$ is some fixed bit string of length $l$).
\\\\
Therefore:
\\\\
$Adv_{F_3}^{PRP}(A_3) = |Pr_K[A_3^{F_3(k, \cdot)} = 1] - Pr_\Pi[A_3^{\pi(\cdot)} = 1]| = |1 - \frac{1}{2^l}| = 1 - \frac{1}{2^l}$

\end{document}

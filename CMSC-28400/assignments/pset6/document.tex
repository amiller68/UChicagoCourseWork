\documentclass[]{article}
\usepackage[ruled,linesnumbered]{algorithm2e}
\usepackage{enumitem}
\usepackage[margin=0.5in]{geometry}
\DontPrintSemicolon

%opening
\title{Problem Set 6}
\author{Alex Miller}

\begin{document}

\maketitle

\begin{abstract}
	Collaborators: Elizabeth Coble, Lucy Li
\end{abstract}

\section{1}
Even if the loop exits after $2^{64}$ iterations, and is therefore computationally feasible for a well equipped adversary, the hash table would by that point take up $2^{64}$ space, which is too large to be considered feasible for even a well equipped adversary. 

\section{2}
\subsection{a}
$h_1(x, v) = E(x, v) \oplus x \\
 h_1(x, v) = E(x, v) \oplus x \oplus 0^B \\
 h_1(x, v) = E(0^B, E^{-1}(0^B, E(x, v) \oplus x)) \oplus 0^B \\
 h_1(x, v) = h_1(x' = 0^B, v' = E^{-1}(0^B, E(x, v) \oplus x))$  

\subsection{c}
$h_3(x, v) = E(0^B, x \oplus v) \oplus x \\
 h_3(x, v) = E(0^B, x \oplus v) \oplus x \oplus 0^B \\
 h_3(x, v) = E(0^B, E^{-1}(0^B,  E(0^B, x \oplus v) \oplus x) \oplus 0^B) \oplus 0^B \\
 h_3(x, v) = h_1(x' = 0^B, v' = E^{-1}(0^B, E(0^B, x \oplus v) \oplus x))$
 
\section{3}
Let $\textbf{B}$ describe the input space of the hash function
\\
\begin{algorithm}[H]
\SetAlgoLined
$m[1] \leftarrow \$ \textbf{B}$\;
$h' \leftarrow E(m[1], k)$\;
\For {$i = 1 ... \sqrt{B} - 1$}{
	$m[2] \leftarrow \$ \textbf{B}$\;
	\If {$h' = E^{-1}(m[2], y)$}{
		return $m[1] || m[2]$
	}
}
return $\bot$
\caption{$A(k, y)$}
\end{algorithm}

$Pr[Expt_H^{pr}(A) = 1] = 1$:
\\\\
Since $E$ and $E^{-1}$ have uniformly random output for a given key, we can define a target intermediary hash $h' = E(m[1], k)$, where $m[1]$ is a random $B$-bit value input and $k$ is a random $B$-bit key. We can use this target value $h'$ to try and find a hood value of $m[2]$ s.t. $h'' = E(m[2], y), h' = h''$. By the birthday principle, since $h'$ and $h''$ are uniformly random integers $\in  \{0,1\}^B$, the probability we generate $h''$ s.t. $h'' = h'$ in $\sqrt{B}$ samples is $\approx \frac{\sqrt{B}^2}{B} = \frac{B}{B} = 1$
\\\\
Therefore $Adv_H^{pr}(A) = Pr[Expt_H^{pr}(A) = 1] = 1$
\\\\
A runs in $2^{\sqrt{B}} < 2^B$ time.
\section{5}

\subsection{c}
Counter Example: $a = 450, b = 300$. $a$ and $b$ are valid for our first relation.
\\\\
$2a = 2b mod 100 \Leftrightarrow 100 | 2(450 - 300) \\
 2a = 2b mod 100 \Leftrightarrow 100 | 300$
\\\\
But these values of $a$ and $b$ do not make it so $a = b mod 100$:
\\\\
$a = b mod 100 \Leftrightarrow 100 | 450 - 300$\\
$a = b mod 100 \Leftrightarrow 100 | 150$
\\\\
but 100 does not divide 150.

\subsection{f}
True:
\\\\
$\exists c \in \textbf{Z} ac = 1 mod N \Leftrightarrow gcd(a, N) = 1 \\
 \exists c \in \textbf{Z} bc = 1 mod N \Leftrightarrow gcd(b, N) = 1 \\$
\\\\
Therefore we can say that neither $a$ nor $b$ share prime factors with $N$; in that case $ab$ should not share prime factors with $N$, so:
\\\\
$gcd(ab, N) = 1 \exists c \in \textbf{Z} abc = 1 mod N$
\\\\
Therefore $ab$ is invertible modulo $N$

\subsection{h}

$(N - 1)$ is invertible modulo $N$:
\\\\
$N - 1$ and $N$ are relatively prime so $gcd(N-1, N) = 1 \Leftrightarrow \exists c \in \textbf{Z} (N - 1)c = 1 mod N$, which means $(N - 1)$ is invertible modulo $N$
\\\\
Since $(N - 1)c = 1 mod N \Leftrightarrow N | (N - 1)c - 1$, we can find a good value for $c$ by considering what would make $N$ a divisor for $(N-1)c - 1$. If $c = (N - 1)$ then $N | (N - 1)(N - 1) - 1 = N^2 - 2N  + 1 - 1 = N^2 -2N$. $N | N^2 - 2N \Leftrightarrow \exists k, N * k = N^2 - 2N$, and in this case $k$ can $= (n - 2)$ (this hold because $N - 2 = k > 0$)
\\\\
Therefore $N - 1$ is the inverse of $N - 1$ modulo $N$
\end{document}
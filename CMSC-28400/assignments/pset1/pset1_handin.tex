\documentclass[]{article}
%opening
\title{Pset 1}
\author{Alex Miller}
\addtolength{\oddsidemargin}{-.875in}
\addtolength{\evensidemargin}{-.875in}
\addtolength{\textwidth}{1.75in}
\addtolength{\topmargin}{-.875in}
\addtolength{\textheight}{1.75in}
\setcounter{secnumdepth}{0}
\usepackage{cancel}
\usepackage{amssymb}
\usepackage{amsmath}
\usepackage{hyperref}
\usepackage{graphicx}
\graphicspath{ {./} }
\usepackage[ruled,vlined, linesnumbered]{algorithm2e}
\DontPrintSemicolon


\begin{document}

\maketitle

I worked with Elizabeth Coble and Lucy Li

\section{Problem 1}
\begin{enumerate}
	\item \
	\begin{enumerate}
		\item $\pi(x) = $ (1 4 6 5 3)
		\item $\sigma(x) = $ (2 6 5 4) (1 3)
	\end{enumerate}
	\item \
	\begin{enumerate}
		\item $\pi^{-1}(x) = $ (4 1 3 5 6)
		\item $\sigma{-1}(x) = $ (2 4 5 6)(1 3)
	\end{enumerate}
	\item \
	\begin{enumerate}
		\item $\pi\sigma(x) = $ (2 5 6 3 4)
		\item $\sigma\pi(x) = $ (1 2 6 4 5)
	\end{enumerate}
\end{enumerate}

\section{Problem 2}
\begin{enumerate}
	\item The order of cycle $A$ = ($a_1$ $a_2$ ... $a_t$) is $t$.
	\begin{itemize}
		\item Consider a vertex $j$ s.t. $1 \leq j \leq t$ on the directed cycle described by $A$. $A$ is a cycle ordering generated by a permutation $\pi$ on the set $\Sigma = \{a_1, a_2, ... , a_t\}$
		\item $A$ has $t$ edges
		\item An observer that wanted to traverse $A$ s.t. they start and end up at $a_j$ would have to traverse $t$ edges in order to do so, and no more.
		\item WLOG, apply the same reasoning to any vertex in $A$. Every vertex on $A$ has a path to itself of length $t$.
		\item Therefore, $\forall \sigma \in \Sigma, \pi^t(\sigma) = \sigma$. $\pi^t(\sigma)$ is the identity of $\Sigma$
		\item Moreover, $t$ is the smallest integer for which this is true.
		\item Therefore $t$ is the order of $A$
		\item \textbf{QED}
	\end{itemize}
	\item Consider a permutation $\pi$ on the set $\Sigma = \{a_1, a_2, ... a_s\}$. $|\Sigma| = s$. $\pi$ is of type $T = [t_1$ $t_2$ ... $t_s]$, where $\forall t \in T$, $t$ represents the number of cycles of $b$ there are on the cyclically ordered graph of $\pi$, and $b \in B = \{1,2,...,s\}$ is $t's$ index in $T$. Let $T(b)$ describe a function that returns the $b$th element of $T$. The order of $\pi$ is equivalent to the least common multiple of the set $\{b \in B : T(b) \geq 1\}$
	\begin{itemize}
		\item Consider the cyclic ordering of $\pi$ on $\Sigma$, $G_{\pi} = (V = \Sigma, E)$
		\item Consider the set of all cycles on $G_{\pi}, C$
		\item $!\exists s, 1 \leq s \leq |\Sigma|$, consider the set $C_s = \{c \subseteq C : |c| = s\}$. If $C_s \neq \emptyset$, the cycles in it are all disjoint cycles on $G_{\pi}$, each with $s$ vertices and edges. By the reasoning employed in part (a), each such cycle is of order $s$. As a corollary, $\pi^s$ is the identity permutation for the set of elements $\bigcup C_s \subseteq \Sigma$
		\item $\forall b \in B,$ if  $1 \leq t = T(b)$, then there is a set $C_b \subseteq C, |C_b| = t$.
		\item The set $A = \{b \in B : 1 \leq T(b)\}$ therefore represents the distinct path lengths of all cycles on $G_{\pi}$
		\item $\forall a \in A, \pi^a$ is an identity for $\bigcup C_a \subseteq \Sigma$
		\item Also, since commuting two identities yields another identity, $\forall 0 < x, \forall a \in A, \pi^{a * x}$ is an identity for $\bigcup C_a \subseteq \Sigma$
		\item Consider some value $m$ for which $\pi^{m}$ is an identity for $\bigcup \{a \in A : \bigcup C_a\} = \Sigma$.
		\item $m$ must be a common multiple of $a \in A$. (Might need to prove)
		\item The smallest possible value of $m$ is the LCM of all $a \in A$
		\item \textbf{QED}
	\end{itemize}
\end{enumerate}

\section{Problem 3}
\begin{itemize}
	\item Consider the permutation $\pi$ on the set $Sigma = \{a_1, a_2, ..., a_t\}$. $\pi$ = ($a_1$ $a_2$ ... $a_t$).
	\item The type of $\pi$ is [$t^1$]. It consists of one cycle of length $t$
	\item The inverse of the cycle described by $\pi$ is $\sigma = $($a_t$ $a_{t-1}$ ... $a_1$).
	\item $\pi\sigma$ and $\sigma\pi$ are identities on $\Sigma$. Applying one and then the other is equivalent to moving forward along one edge, and then backwards along the same edge on a graph describing $\Sigma$
	\item Therefore $\sigma = \pi^{-1}$.
	\item The type of $\pi^{-1}$ is [$t^1$]. It consists of one cycle of length $t$
	\item $\pi$ and $\pi^{-1}$ have the same type
	\item \textbf{QED}
\end{itemize}

\section{Problem 4}
The components of $G_f$ will all contain one cycle, and may contain external directed paths leading into this cycle. In other words, every vertex has at least one out going edge, and zero or more incoming edges.

\section{Problem 5}
All reflectors permutations are type $[2^{13}]$. This is equivalent to saying that a reflector permutation encodes 13 distinct pairs on $\Sigma = \{A, B, C, ..., D\}$ (the letters of the English alphabet).Therefore, the number of distinct pair-wise groupings on the letters of the alphabet is equivalent to the number of distinct reflector permutations on it. To find this figure:
\begin{enumerate} 
	\item Consider all possible permutations of the letters in the alphabet. There are $26!$ of these.
	\item Consider one such permutation, call it $p$. We can generate a grouping of the elements of this permutation by pairing the first and second elements, the third and four elements, ..., and the 25th and 26th elements. Call this grouping $g$
	\item We observe that swapping the position of groupings in $g$ yields the same set of groupings. Additionally, we know that for an alphabet of 26 letters, there are $13!$ distinct ways to order the elements of $g$.
	\item We further observe that for all groupings, there are $2$ ways to represent the same grouping with its elements. For example, the grouping $\{A, b\} = \{B, A\}$. Therefore, given a grouping $g$ with 13 elements, there are $2^{13}$ ways to rearrange the contents of pairings on $g$ (without messing with the order of $g$) 
	\item Putting it all together: for the English alphabet, we can generate $26!$ permutations, each of which represent $13!$ groupings. Each grouping is equivalent to $2^{13}$ distinct orderings of the elements in the grouping.
	\item Therefore there are $\frac{26!}{2^{13}13!}$ ways to distinctly group the letters in the alphabet.
	\item There are $\frac{26!}{2^{13}13!}$ distinct reflector permutations.
	\item \textbf{QED}
\end{enumerate}


\end{document}

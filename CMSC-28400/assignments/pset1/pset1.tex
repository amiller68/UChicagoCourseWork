\documentclass[11pt]{article}
\usepackage{fullpage}
\usepackage{amsthm}

\usepackage{amsthm,amsmath,amsfonts,amssymb,amstext,enumitem}
\usepackage{latexsym,ifthen,url,rotating}

\usepackage[usenames,dvipsnames]{color}

% --- -----------------------------------------------------------------
% --- Document-specific definitions.
% --- -----------------------------------------------------------------

\newtheorem{problem}{Problem}
\newtheorem{theorem}{Theorem}
\newtheorem{definition}{Definition}
\newtheorem{lemma}{Lemma}

\newcommand{\getsr}
  {{\:\stackrel{\raisebox{-2pt}{${\scriptscriptstyle \hspace{0.2em}\$}$}}
   {\leftarrow}\:}}
\newcommand{\points}[1]{\textbf{({#1} pts)}}

\newcommand{\Colon}{\ : \ }
\newcommand{\st}{\mathsf{state}}
\newcommand{\msgsp}{\mathcal{M}}
\newcommand{\keys}{\mathcal{K}}
\newcommand{\kg}{\mathcal{K}}
\newcommand{\enc}{\mathcal{E}}
\newcommand{\decc}{\mathcal{D}}
\newcommand{\MAC}{\mathrm{MAC}}
\newcommand{\RMAC}{\mathrm{RMAC}}

\newcommand{\pk}{pk}
\newcommand{\sk}{sk}

\newcommand{\AES}{\mathsf{AES}}

\newcommand{\algorithm}[1]{\textbf{Alg} {#1}}

\newcommand{\calO}{\mathcal{O}}

\newcommand{\dlog}{\mathrm{dlog}}

\newcommand{\Adv}{\mathbf{Adv}}
\newcommand{\AdvPRF}[2]{\Adv^{\mathrm{prf}}_{#1}({#2})}
\newcommand{\AdvCPA}[2]{\Adv^{\mathrm{ind{-}cpa}}_{#1}({#2})}
\newcommand{\AdvCCA}[2]{\Adv^{\mathrm{ind{-}cca}}_{#1}({#2})}
\newcommand{\AdvKR}[2]{\Adv^{\mathrm{kr}}_{#1}({#2})}
\newcommand{\AdvCKR}[2]{\Adv^{\mathrm{ckr}}_{#1}({#2})}
\newcommand{\AdvRMR}[2]{\Adv^{\mathrm{rmr}}_{#1}({#2})}
\newcommand{\AdvCR}[2]{\Adv^{\mathrm{cr}}_{#1}({#2})}
\newcommand{\AdvUFCMA}[2]{\Adv^{\textrm{uf{-}cma}}_{#1}({#2})}
\newcommand{\AdvDL}[2]{\Adv^{\mathrm{dl}}_{#1}({#2})}

\newcommand{\Exp}{\mathbf{Exp}}
\newcommand{\ExpOW}[1]{\Exp^{\mathrm{ow}}({#1})}
\newcommand{\ExpCKR}[2]{\Exp^{\mathrm{ckr}}_{#1}({#2})}
\newcommand{\ExpRMR}[2]{\Exp^{\mathrm{rmr}}_{#1}({#2})}

\newcommand{\concat}{{\,\|\,}}
\newcommand{\xor}{\oplus}
\newcommand{\bits}{\{0,1\}}

\newcommand{\tcolh}{T^{\mathrm{col}}_h}
\newcommand{\tcolH}{T^{\mathrm{col}}_{H^2}}
\newcommand{\Hcomb}{H^{1\|2}}
\newcommand{\Hxor}{H^{1\oplus2}}

\newcommand{\EXP}{\textrm{EXP}}
\newcommand{\MODEXP}{\textrm{MOD{-}EXP}}
\newcommand{\ADD}{\textrm{ADD}}
\newcommand{\MULTIMODEXP}{\textrm{MULTI{-}MOD{-}EXP}}
\newcommand{\MUL}{\textrm{MUL}}
\newcommand{\MOD}{\textrm{MOD}}

\newcommand{\GG}{\mathbb{G}}
\newcommand{\ZZ}{\mathbb{Z}}

% --- -----------------------------------------------------------------
% --- The document starts here.
% --- -----------------------------------------------------------------
\begin{document}
\sloppy

\noindent 
CMSC 28400: Introduction to Cryptography, Autumn 2021\\
University of Chicago\\
Course Staff: Akshima, David Cash (instructor), Andrew Chu, Alex Hoover

\begin{center}
\LARGE{\textbf{Problem Set 1}}\\
    \large{Due in Thursday, October 7 at 11:59pm}
\end{center}

\vspace{.1in}

\begin{description}

    \item[Collaboration policy.] Please respect the following collaboration
        policy:  You may discuss problems together in groups of up to four
        \emph{but you must write up your own solutions. You should never see
        your collaborators' papers or code.}  At the beginning of your
        submission, indicate the names of your (1, 2, or 3) collaborators, if
        any.   You may switch groups between problem sets but not within the
        same problem set.

    \item[Sources.] Cite any sources you use. Citing lecture, providing
    readings, or the other free textbooks linked from the course syllabus
    webpage is allowed. You may use results proved in those sources or in
    lecture without repeating their proofs.  Using Google, Chegg, or searching
    for posted solutions from other universities, is not allowed.

    \item[Grading.] Responses to theory problems will be graded for
        correctness, precision, \emph{and clarity}.  Simple, well-explained
        proofs are preferred.  Point values of problems vary, and are listed.  
        You are encouraged to ask for help and advice from the staff in writing
        up your solutions.

    \item[Submitting your solutions.] Solutions will be accepted on Gradescope.
    Check Campuswire for instructions.

    You may submit well-lit, clear photographs of handwritten solutions. You may
    also type up your solutions by modifying the included tex file -- The course
    staff will be happy to provide support if you choose to learn the Latex
    typesetting language.

\end{description}
\hrulefill

\begin{enumerate}

    \item \points{25} Warm-up: Let $\Sigma = \{1,2,3,4,5,6\}$, and let
        $\pi$ and $\sigma$ be permutations on $\Sigma$ defined by

        \begin{center}
        \begin{tabular}{cc}
            \begin{minipage}{2in}
                \begin{tabular}{c|*{7}c}
                    $x$      & 1 & 2 & 3 & 4 & 5 & 6 \\ \hline
                    $\pi(x)$ & 4 & 2 & 1 & 6 & 3 & 5
                \end{tabular}
            \end{minipage}
            and
            &
            \begin{minipage}{2in}
                \begin{tabular}{c|*{7}c}
                    $x$         & 1 & 2 & 3 & 4 & 5 & 6 \\ \hline
                    $\sigma(x)$ & 3 & 6 & 1 & 2 & 4 & 5
                \end{tabular}.
            \end{minipage}
        \end{tabular}
        \end{center}
        For each part here, you don't need to show any work.
        \begin{itemize}
            \item Write each of $\pi$ and $\sigma$ as products of disjoint
                cycles.

            \item Find $\pi^{-1}$ and $\sigma^{-1}$, written as products
                of disjoint cycles.

            \item Find $\pi\sigma$ and $\sigma\pi$, written as products
                of disjoint cycles.
        \end{itemize}

    \item \points{50} Define the \emph{order of a permutation $\pi$} to be the
        minimum
        $k\geq 1$ such that $\pi^k$ is the identity, where $\pi^k$ means
        $\pi$ composed with itself $k$ times.
        \begin{enumerate}
            \item Find the order of a cycle $(a_1\ a_2\ \cdots\ a_t)$.
            (Explain.)
            \item Find the order of an arbitrary permutation $\pi$ 
                in terms of its type. (Explain.)
        \end{enumerate}

    \item \points{50} Find the inverse of a cycle $(a_1\ a_2\ \cdots\ a_t)$.
        Use this (or another approach if you prefer) to prove that $\pi^{-1}$
        has the same type as $\pi$.

    \item \points{25} We have seen that there is a one-to-one correspondence
        between permutations $\pi$ on a set $\Sigma$ and cycle-partition graphs
        $G_\pi$ on the vertex set $V=\Sigma$.  Consider now directed graphs on
        $V=\Sigma$ that correspond to arbitrary functions $f:\Sigma\to\Sigma$
        (as before, for each $x\in\Sigma$ there is a directed edge from $x$ to
        $f(x)$).
            
        The components of $G_f$ will not be directed cycles when $f$ is not
        a permutation, but they cannot be arbitrary directed graphs either. 
        Give a description of the components of $G_f$. [You don't need to prove
        your answer here; a one-sentence description is fine.]

        We will see more of the graphs $G_f$ later in the course.

    \item \points{50} Rejewski states that the number of reflector permutations
        on a set $\Sigma$ of size $26$ is
        \[
            \frac{26!}{2^{13}13!}.
        \]
        Prove this formula by explaining how it counts these permutations.


\end{enumerate}

\end{document}

